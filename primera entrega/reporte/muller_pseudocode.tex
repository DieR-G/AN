\section{Pseudocódigo.}
El siguiente algoritmo asume la utilización de aritmética de números complejos dentro del mismo,
además la operación de diferencias divididas calculada de forma externa con un procedimiento similar
al mostrado en (\refeq{newton_div}). La condición de paro es que el valor absoluto de la función evaluada
en la última aproximación encontrada sea menor a una tolerancia $\epsilon$ o que la cantidad de iteraciones
supere un valor límite previamente establecido.
\begin{algorithm}
    \DontPrintSemicolon
    \KwData{$x_2,x_1,x_0$ valores distintos para interpolar $f(x)$, un valor $\epsilon$ de tolerancia y un valor $N$
    que define la cantidad máxima de iteraciones.}
    \KwResult{La raíz de $f(x)$ más cercana a los puntos iniciales dados o un mensaje de error en
    caso de que no se cumpla la tolerancia mínima}
    \Begin{
    $x_{n+1} \gets 0$\;
    $x_{n} \gets x_2$\;
    $x_{n-1} \gets x_1$\;
    $x_{n-2} \gets x_0$\;
    $x_{n+1} \gets 3$\;
    \While{$|f(x_{n+1})|<\epsilon \,{\bf or}\, i \geq N $}{
        $a \gets f[x_n,x_{n-1},x_{n-2}]$\;
        $b \gets f[x_n,x_{n-1}]+f[x_n,x_{n-2}]-f[x_{n-1},x_{n-2}]$\;
        $c \gets f(x_n)$\;
        $D \gets \sqrt{b^2-4ac}$\;
        \tcc{Escoge el valor de mayor magnitud para el denominador}
        \eIf(){$|b-D|\le|b+D|$}{
            $E \gets b+D$
        }(){$E \gets b-d$}
        $x_{n+1} \gets x_n-\frac{2c}{E}$\;
        $x_{n-2} \gets x_{n-1}$\;
        $x_{n-1} \gets x_{n}$\;
        $x_{n} \gets x_{n+1}$\;
        $i \gets i+1$
    }
    \eIf{$|f(x_{n+1})|<\epsilon$}{
        \Return{$x_{n+1}$}
    }
    {\KwOut{No se pudo encontrar un valor para la raíz.}}
    }
    \caption{Método de Muller\label{IR}}
\end{algorithm}