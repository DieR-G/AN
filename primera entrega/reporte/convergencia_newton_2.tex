\subsection{Convergencia para dos raíces o más}

Habiendo observado como el método de Newton converge de manera muy rápida hacia la raíz cuando existe solamente una, podemos extender la anterior definición para emplear el método a n raíces, con $n \geq 2$, partiendo de la premisa que todo $p_0$ dentro de $[p-\delta, p+\delta]$ convergerá al valor de la raíz más cercana, siempre y cuando dentro del intervalo escogido se cumplan las condiciones de que $f,f'$ son continuas y $f'(p_0) \neq 0$, y por lo tanto podemos enunciar la siguiente definición

\begin{definition}

Si $p_*$ es una raíz de \textit{f}, el área de atracción de $p_*$ esta compuesta por todos aquellos valores $p_0$ tales que el método de Newton al comenzar desde $p_0$ converge hacia $p_*$

\begin{displaymath}
B(x_*) = {x_0|x_n =  N^n(x_0) \xrightarrow{}  x_*}
\end{displaymath}

\end{definition}

A partir de la determinación del tipo de puntos al que pertenecen los valores de $p_0$, podremos establecer que los puntos iniciales tenderán a la raíz más cercana. 
Los tipos de puntos fijos se establecen en base a la siguiente definición:

\begin{definition}
 Supóngase un mapa $f: X \xrightarrow[]{}X$ es diferenciable en un punto fijo $p_*$, entonces
 \begin{itemize}
     \item $p_*$ es atrayente si y solo si $|f(p_*)|<1$
     \item $p_*$ es super-atrayente si y solo si $|f(p_*)|=0$
     \item $p_*$ es repelente si y solo si $|f(p_*)|>1$
 \end{itemize}
\end{definition}

Entonces, en vista de que $|g'(p)| = 0$ ya que 
\begin{displaymath}
     g(x)= 1 - \frac{f(x)}{f'(x)} 
\end{displaymath}

con derivada 

\begin{displaymath}
     g'(x)= 1 - \frac{(f'(x)f'(x))-(f(x)f''(x))}{(f'(x))^2} = \frac{(f(x)f''(x))}{(f'(x))^2}
\end{displaymath}

y asumiendo que $f(p)=0$
\begin{displaymath}
     g'(x)= 0
\end{displaymath}

lo que permite concluir que los puntos dentro del intervalo $[p-\delta, p+\delta]$, son super-atrayentes y por ende convergerán a la raíz más cercana

Este comportamiento puede observarse buscando las raíces de la ecuación $x^2-1 = 0$, la cual posee dos raíces reales 1 y -1 como puede verse en Figura \ref{fig:graph_cuadratica}. 

\begin{figure}[H]
    \begin{tikzpicture}
\begin{axis}[
axis x line = middle,
axis y line =  center,
grid = minor
]
\addplot[
color=red,
]{x^2-1};
\addplot[only marks] table {
1   0
-1  0
};
\end{axis}
\end{tikzpicture}
\centering
    \caption{Gráfica de $x^2-1$}
    \label{fig:graph_cuadratica}
\end{figure}


Coloreando cada $p_0$ en base a que raíz se acercarán en la aproximación, vemos que obtenemos la siguiente imagen

Donde se observa como el plano se divide en dos partes uniformes con su respectivo color, en base a cual será la raíz a la que el valor se aproximará

\begin{figure}[h]
    \centering
    \includegraphics{images/eq1-1.png}
    \caption{Zonas de convergencia de $x^2-1$}
    \label{fig:eq_cuadratica_1}
\end{figure}

si, ahondamos más en la imagen y generamos un gradiente de color, donde mientras más oscuro más rápidamente converge el punto, o lo que es lo mismo, menos iteraciones se necesitan, obtenemos lo siguiente:
\begin{figure}[H]
    \centering
    \includegraphics[scale=0.26]{images/eq1-2.png}
    \caption{Rapidez de convergencia de los puntos de $x^2-1$}
    \label{fig:eq_cuadratica_2}
\end{figure}


Donde podemos ver que al aproximarnos a 0, los ritmos de convergencia tienden a aumentar de manera drástica, al punto de no converger.
Esto se debe a que si evaluamos un valor de $p_n = 0$, $g'(0) = 2x = 2(0) = 0$, indefiniendo el resultado del método en ese punto

Veamos otro ejemplo con un comportamiento curioso, el cual también tiene puntos donde el método se indefine
Sea $f(x) = x^3-x$, que al resolver algebraicamente obtenemos $(x-1)(x-0)(x+1)$ con raices en 1,0,-1.
A partir de su gráfica vemos que existen dos puntos críticos, que determinándolos a través de $f'(x) = 0$ da como resultado $-\frac{1}{\sqrt{3}}, \frac{1}{\sqrt{3}}$. 

\begin{figure}[h]
    \begin{tikzpicture}
\begin{axis}[
axis x line = middle,
axis y line =  center,
grid = minor
]
\addplot[
color=red,
]{x^3-x};
\addplot[only marks] table {
1   0
-1  0
0   0
};
\end{axis}
\end{tikzpicture}
\centering
    \caption{Gráfica de $x^3-x$}
    \label{fig:graph_cubica}
\end{figure}


De esto podemos concluir que $(-\infty,-\frac{1}{\sqrt{3}}) \subset  B(-1)$ y $(\frac{1}{\sqrt{3}}, \infty) \subset  B(1)$. Para el caso de $B(0)$, su determinación es un poco más complicada, ya que debemos tomar en cuenta los puntos crítico, los cuales producirán un efecto cíclico, en este caso de período 2, en Newton-Raphson si se les tome como puntos iniciales o cualquier punto inicial que, después de n iteraciones, haga que la tangente de la función sea igual a uno de ellos \cite{wiersma}. 

Partiendo la premisa anterior premisa, tenemos que determinar en que punto $x_0 = x_2$, es decir $x = N(N(x)) = N^2(x)$.  Este proceso se simplifica ya que $f$ es una función impar y entonces $-f(x) = f(-x)$, por lo tanto a través de simetría tenemos que obtendremos $N^2(x)$ si $-x = N(x)$. Para esto calculamos la función en el método

\begin{gather*}
    N(x) = x - \frac{x^3-x}{3x^2-1} \\
    N(x) = \frac{2x^3}{3x^2-1}
\end{gather*}

luego igualamos a -x y resolvemos

\begin{gather*}
    -x = \frac{2x^3}{3x^2-1}\\
    0 = 5x^3-x\\
    x = \pm \frac{1}{\sqrt{5}} \vee x = 0
\end{gather*}

Y de esto podemos concluir que $(-\frac{1}{\sqrt{5}},\frac{1}{\sqrt{5}}) \subset B(0)$
\\
\\
Ahora bien, veamos que pasa entre los valores de $(-\frac{1}{\sqrt{3}},-\frac{1}{\sqrt{5}})$ y $(\frac{1}{\sqrt{5}},\frac{1}{\sqrt{3}})$. Como la función es simétrica, entonces lo que pasa en $(\frac{1}{\sqrt{5}},\frac{1}{\sqrt{3}})$, será igual que en $(-\frac{1}{\sqrt{3}},-\frac{1}{\sqrt{5}})$. En estos dos intervalos se suceden áreas de convergencia hacia $B(1)$ y $B(-1)$, las cuales se ven divididas por valores los cuales luego de n iteraciones caerían en alguno de los puntos críticos de la función, indefiniendo el método.



Veamos por ejemplo un punto un poco a la izquierda de $\frac{1}{\sqrt{3}}$, por ejemplo 0.5773442692, el cual si evaluamos en Newton, nos devolverá $-18518.03739$, un valor que se encuentra dentro de los puntos convergentes hacia -1, hasta que encontremos un valor cuya n iteración sea $-\frac{1}{\sqrt{3}}$ , el cual al igualar ese valor en el método obtenemos 0.465601, por lo tanto $(\frac{1}{\sqrt{3}},0.465601) \subset  B(-1)$.
Si luego probamos con 0.465600, la siguiente iteración devuelve un número positivo muy grande, que cae dentro de $B(1)$. Este comportamiento se mantendrá hasta que hallemos un número cuya n-esima iteración de como resultado $\frac{1}{\sqrt{3}}$ e indefina la función. 

Estos patrones continúan y se alternan entre cada división causada por valores que iteren hacia alguno de los puntos críticos de la función, y podremos determinar cada una de estas divisiones igualando l a función a uno de los puntos previos obtenidos,, esto nos genera unos patrones curiosos, como puede verse en Figura \ref{fig:eq_cubica_1} y \ref{fig:eq_cubica_2}
\begin{figure}[H]

        \centering
        \includegraphics{images/eq2-1.png}
        \caption{Áreas de convergencia de $x^3-x$}
        \label{fig:eq_cubica_1}
\end{figure}
    
\begin{figure}[H]
    \centering
    \includegraphics[scale=0.26]{images/eq2-2.png}
    \caption{Rapidez de convergencia de los puntos de $ x^3-x$}
    \label{fig:eq_cubica_2}
\end{figure}
