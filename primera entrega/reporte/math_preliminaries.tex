\section{Preliminares matemáticos.}
\subsection{Polinomio interpolante.}
Si se conoce el valor de una función $f(x)$ en n+1 puntos defínase el polinomio
interpolante como un polinomio de grado n que pasa por los n+1 puntos
previamente establecidos y se utiliza para aproximar los valores de $f(x)$ en las cercanías de cada punto.
La deducción más detallada de cómo se puede encontrar dicho polinomio y toda la teoría detrás
se salen del ámbito de esta investigación, sin embargo, una explicación puede encontrarse en cualquier texto
sobre análisis y métodos numéricos (véase por ejemplo \cite{atkinson}).
\subsection{Polinomio interpolante de Newton.}
\begin{definition}
    \label{newton_poly}
    Dado un conjunto distinto de puntos $(x_n,f(x_n)),\ldots,(x_0,f(x_0))$ se define
    el polinomio interpolante de Newton como un polinomio de grado n de la forma
    \begin{equation}
        \begin{split}
            &P(x) =f[x_n]+f[x_n,x_{n-1}](x-x_n)\\
            &\quad +f[x_n,x_{n-1},x_{n-2}](x-x_n)(x-x_{n-1})\\
            &\qquad + \cdots + f[x_n,\ldots,x_0](x-x_n)(x-x_{n-1})\cdots(x-x_1) 
        \end{split} 
    \end{equation}  
\end{definition}
donde $f[x_n,\ldots,x_i]$ es la operación conocida como diferencia dividida de Newton,
la cual se definirá más adelante. Del anterior planteamiento es posible observar cómo
los n+1 puntos acaban definiendo un polinomio de grado n que sirve para interpolar
la función en los valores conocidos.
\subsection{Cota de error de un polinomio interpolante}
Considerando entonces que un polinomio interpolante aproxima una función 
en las cercanías de los n+1 puntos que lo definen, es posible estimar cuál es
la magnitud máxima del error que se comente al realizar tal aproximación. Sea $P(x)$ el
polinomio interpolante y $f(x)$ una función con n+1 derivadas continuas definidas en el menor intervalo $I$ que contenga los números reales $[x_n,x_{n-1},\ldots,x_0]$,
se define el error cometido como la expresión
\begin{equation}
    \label{lagrange_error}
    f(x)-P(x)=\frac{f^{\left(n+1\right)}(\xi)}{(n+1)!}(x-x_n)\cdots(x-x_0)
\end{equation}
con $\xi \in I$. \\De nuevo la demostración de la anterior expresión
va más allá del ámbito de este trabajo y puede encontrarse en cualquier
libro de análisis numérico.
\subsection{Diferencias divididas.}
Dada una función $f(x)$ se definen sus diferencias divididas como una
operación recursiva de la forma
\begin{align*}
    f[x_n]                      &= f(x_n)\\
    f[x_n,x_{n-1}]              &= \frac{f[x_{n-1}]-f[x_n]}{x_{n-1}-x_n}\\
                                &\phantom{=} \vdots\\
    \numberthis \label{newton_div}f[x_n,\ldots,x_0] &= \frac{f[x_{n-1},\ldots,x_0]-f[x_{n},\ldots,x_1]}{x_0-x_n} 
\end{align*}
Donde $f[x_n,x_{n-1}]$ es la primera diferencia dividida, $f[x_n,x_{n-1},x_{n-2}]$
es la segunda diferencia dividida y así sucesivamente.

\begin{lemma}
    \label{diff_deriv}
    $f[x_n,\ldots,x_0]=\frac{f^n(\xi)}{n!}$ para algún $\xi \in I$, donde $I$ es el intervalo
    más pequeño que contiene los números reales $[x_n,\ldots,x_0]$.
    \begin{proof}
        Primero se plantea una función $f(x) \in C^nI$ aproximada por un polinomio interpolante
        $P(x)$ de grado n partiendo de n+1 puntos $[x_n,\ldots,x_0]$ establecido anteriormente
        en la definición (\ref{newton_poly}). Dado que $P(x)$ es una interpolación de $f(x)$ se plantea la
        función $g(x)=f(x)-P(x)$ la cual tiene n+1 raíces, así que por el teorema de Rolle
        generalizado se tiene que:
            \begin{equation}
                \begin{split}
                    g^n(\xi)&=f^n(\xi)-P^n(\xi)=0\\
                    g^n(\xi)&=f^n(\xi)-f[x_n,\ldots,x_0]n!=0\\
                    f[x_n,\ldots,x_0] &= \frac{f^n(\xi)}{n!}
                \end{split}
            \end{equation}
    \end{proof}
\end{lemma}