\documentclass[9pt,journal]{IEEEtran}
\usepackage[spanish]{babel}
\usepackage[utf8]{inputenc}
\usepackage[spanish]{babel}
\decimalpoint
\usepackage{mathtools}
\usepackage{amsmath,amsthm,amsfonts,amssymb}
\usepackage{tikz}
\usepackage{pgfplots}
\pgfplotsset{width=10cm,compat=1.18}
\usepackage[linesnumbered,ruled,boxed,commentsnumbered]{algorithm2e}

\newtheorem{lemma}{Lemma}[section]
\newtheorem{theorem}{Teorema}[section]
\newtheorem{definition}{Definición}[section]
\numberwithin{equation}{section}

\title{Método de posición falsa (Regula Falsi)}
\author{ERNESTO JOSE CANALES GUILLEN}
\date{May 2022}

\begin{document}

\maketitle

\section{El método de posición falsa (Regula Falsi)}

El método de la falsa posición pretende conjugar la seguridad del método de la bisección con la rapidez del método de la secante.\newline\newline
Como es denominado \cite{Burden_English}, el término Regula Falsi literalmente “regla falsa” o “posición falsa” (a veces es denominado como el método de adivinar y comprobar. \cite{CS_Web}) se refiera a una técnica en la que se usan resultados que se sabe son falsos, pero de algún modo específico, genera aproximaciones de la misma manera que el método de la secante, pero incluye una prueba para garantizar que la raíz siempre se agrupa entre iteraciones sucesivas, para obtener convergencia a un resultado verdadero. Aunque su convergencia está garantizada, tiene una tasa de convergencia lenta.\newline\newline
Los problemas de posición falsa se pueden encontrar en el papiro Rhind, que data de aproximadamente 1650 a.C.

\begin{figure}[!ht]
    \centering
    \begin{tikzpicture}
        \begin{axis}[
            axis x line = middle,
            axis y line =  center,
            grid = minor,
            xticklabels={,,},
            yticklabels={,,}
            ]
            \addplot[
                color = red,
                domain=0.5:3.5
                ]{x - 2*sin(deg(x)) - (1 / 2)};
            \addplot[
                color = black,
                domain=0.5:3.5
                ]{(((2.2177 - -1.1829) * (x - 1)) / (3 - 1)) + -1.1829};
            \addplot[
                color = black,
                domain=0.5:3.5
                ]{(((2.2177 - -0.788714) * (x - 1.6957)) / (3 - 1.6957)) + -0.788714};
            \filldraw[black] (2.1613,-0.0000037) circle (2pt) node[anchor=north west]{Raíz};
            \filldraw[black] (1, -1.1829) circle (2pt) node[anchor=west]{A};
            \filldraw[black] (3, 2.2177) circle (2pt) node[anchor=west]{B};
            \filldraw[black] (1.6957, 0) circle (2pt) node[anchor=west]{};
            \filldraw[black] (2.0378, 0) circle (2pt) node[anchor=west]{};
        \end{axis}
    \end{tikzpicture}
    \caption{Gráfica del método de la falsa posición}
    \label{regula_grap_ej1}
\end{figure}
\subsection{Deducción del método}

Originalmente en \cite{AN_Spanish}, tenemos que si se pueden elegir dos aproximaciones iniciales $ x_{n-1} $ y $ x_n $ tales que los dos valores de la función de esos puntos tengan signo opuesto en una función continúa comprendida entre $ [x_{n-1}, x_n] $ ], entonces es posible generar una sucesión de valores que siempre tengan esta propiedad.\newline\newline
Para iniciar, construimos la recta que pasa por los puntos $ (x_{n-1}, f(x_{n-1})) $ y $ (x_n, f(x_n) $. De acuerdo con la figura, se tiene que $ m_1 = m_2 $;\newline\newline

%Grafica

Se calcula la intersección con el eje $ x $ de la recta trazada anteriormente y a este punto se le denotará como $ x_{n+1} $.\newline\newline
Para encontrar la ecuación general que nos dará $ x_{n+1} $ en cada recta trazada, primero necesitamos encontrar la pendiente de la recta $ m_2 $:

\begin{displaymath}
    m_2 = \frac{f(x_n) - f(x_{n-1})}{x_n - x_{n-1}}
\end{displaymath}

Luego encontramos la pendiente de la recta $ m_1 $ que quedaría desde el intercepto hasta el extremo del intervalo en donde la función cambia de signo:

\begin{displaymath}
    m_1 = \frac{f(x_{n+1}) - f(x_{n-1})}{x_{n+1} - x_{n-1}}
\end{displaymath}

Por tanto, igualando las ecuaciones de ambas pendientes, debido que son dos segmentos de la misma recta:

\begin{displaymath}
    m_1 = m_2
\end{displaymath}

\begin{displaymath}
    \frac{f(x_{n+1}) - f(x_{n-1})}{x_{n+1} - x_{n-1}} = \frac{f(x_n) - f(x_{n-1})}{x_n - x_{n-1}}
\end{displaymath}

El valor del cruce por cero se define cuando se tiene un valor de $ x_{n+1} $, dado por la recta definida por la ecuación anterior, donde se cumple que $ f(x_{n+1}) = 0 $.\newline\newline
Así, la ecuación anterior queda de la siguiente forma:

\begin{displaymath}
    \frac{0 - f(x_{n-1})}{x_{n+1} - x_{n-1}} = \frac{f(x_n) - f(x_{n-1})}{x_n - x_{n-1}}
\end{displaymath}

Despejando $ x_{n+1} $ se obtiene:

\begin{displaymath}
    (x_n - x_{n-1})(0 - f(x_{n-1})) = (x_{n+1} - x_{n-1})(f(x_n) - f(x_{n-1}))
\end{displaymath}

\begin{displaymath}
    -\frac{(x_n - x_{n-1})(f(x_{n-1}))}{(f(x_n) - f(x_{n-1}))} = (x_{n+1} - x_{n-1})
\end{displaymath}

\begin{equation}
    \label{eq1_regula_falsi}
    x_{n+1} = x_{n-1} - \frac{(x_n - x_{n-1})(f(x_{n-1}))}{(f(x_n) - f(x_{n-1}))}
\end{equation}

Simplificando \eqref{eq1_regula_falsi}:

\begin{equation}
    \label{eq2_regula_falsi}
    x_{n+1} = \frac{(x_{n-1})(f(x_n)) - (x_n)(f(x_{n-1}))}{(f(x_n) - f(x_{n-1}))}
\end{equation}

Utilizando la ecuación \eqref{eq2_regula_falsi} anterior, el valor de $ x_{n+2} $ se elige tomando un valor entre $ x_{n-1} $ y $ x_n $ de tal forma que el valor de la función sea opuesto en signo a $ f(x_{n+1}) $. Así, valores de $ f(x_{n+1}) $ y $ x_{n+2} $ definen un menor intervalo que contiene el cruce por cero. El proceso continúa tomando siempre lados opuestos del cruce por cero. 
\section{Demostración de convergencia y análisis del error}

\subsection{Preliminares a tomar en cuenta \cite{MR_PDF}:}

Para la velocidad de convergencia, consideramos una secuencia $ [q_0,q_1,..] $ converge en $ q $, luego la diferencia $ (q_n - q) $ debe ser menor a medida que $ n $ se acerca al infinito; tomando la diferencia $ (q_n - q) $ como error $ (e_n) $, es dada por $ e_n = q_n - q $.\newline\newline
Tomando la proporción de errores sucesivos $ \frac{|e_{n+1}|}{|e_n|} $ queremos que sea menor que uno, es decir, si $ n $ tiende al infinito, la proporción debe ser menor que uno.

\begin{displaymath}
    \lim_{n^{-} \to \infty} (\frac{|e_{n+1}|}{|e_n|}) = k < 1
\end{displaymath}
	
Entonces el orden de convergencia es uno o la secuencia converge linealmente.\newline
Generalmente, tenemos:

\begin{displaymath}
    \lim_{n^{-} \to \infty} (\frac{|e_{n+1}|}{|e_n|^\alpha}) = k < 1
\end{displaymath}

\subsection{Tasa de convergencia:}

En el método de posición falsa, la secuencia de aproximación es $ {x_1,x_2,\ldots x_n} $ que converge en $ x $. Sea $ e_n $: el error se da como $ x_n - x $. La sucesión converge si $ |e_n| \rightarrow 0 $ como $ |n| \rightarrow \infty $. El orden de convergencia está determinado por una relación asintótica $ |e_n| $ y $ |e_n - 1| $. Ahora bien, si consideramos una función $ f(x)=0 $ en el intervalo $ (x_0, x_1) $ que contiene la raíz, luego en el método de posición falsa, uno de los dos puntos dados $ x_0 $ o $ x_1 $ siempre son fijos y otros varían. Si se fija el punto $ x_0 $, la función se aproxima mediante una recta que pasa por $ (x_0, f(x_0)) $ y $ (x_n, f(x_n)) $, donde $ n=1,2,\ldots $

% grafica

\subsection{Calcular el error:}

Sea $ {x_n} \rightarrow x $. Cuando $ x_n $ es una secuencia $ {x_0,x_1,x_2,\ldots} $ converge a $ x $, entonces el error en la iteración $ n_{th} $ es dado por $ e_n = x_n - x $ para $ n > 0 $.\newline\newline
Sea $ \lambda $ y $ \alpha $ constantes positivas tal que:

\begin{displaymath}
    \lim_{n^{-} \to \infty} (\frac{|x_{n+1} - x|}{|x_n - x|^\alpha})
\end{displaymath}

\begin{displaymath}
    \lim_{n^{-} \to \infty} (\frac{|e_{n+1}|}{|e_n|^\alpha}) = \lambda
\end{displaymath}

Sabemos por la formula de la falsa posición \eqref{eq2_regula_falsi}:

\begin{displaymath}
    x_{n+1} = \frac{(x_{n-1})(f(x_n)) - (x_n)(f(x_{n-1}))}{(f(x_n) - f(x_{n-1}))}
\end{displaymath}

Podemos reescribir la ecuación \eqref{eq2_regula_falsi} como:

\begin{equation}
    \label{eq3_regula_falsi}
    x_n = \frac{(x_{n-2})(f(x_{n-1})) - (x_{n-1})(f(x_{n-2}))}{(f(x_{n-1}) - f(x_{n-2}))}
\end{equation}

Además, calculamos sus errores como:

\begin{equation}
    \label{eq4_regula_falsi}
    e_n = x_n - x \rightarrow x_n = e_n + x
\end{equation}

\begin{equation}
    \label{eq5_regula_falsi}
    e_{n-1} = x_{n-1} - x \rightarrow x_{n-1} = e_{n-1} + x
\end{equation}

\begin{equation}
    \label{eq6_regula_falsi}
    e_{n-2} = x_{n-2} - x \rightarrow x_{n-2} = e_{n-2} + x
\end{equation}

Sustituyendo los valores de \eqref{eq4_regula_falsi}, \eqref{eq5_regula_falsi} y \eqref{eq6_regula_falsi} en la ecuación \eqref{eq3_regula_falsi} obtenemos:

\begin{displaymath}
    e_n + x = \frac{(e_{n-2} + x)(f(e_{n-1} + x))-(e_{n-1} + x)(f(e_{n-2} + x))}{f(e_{n-1} + x) - f(e_{n-2} + x)}
\end{displaymath}

\begin{displaymath}
    e_n = \frac{(e_{n-2} + x)(f(e_{n-1} + x))-(e_{n-1} + x)(f(e_{n-2} + x))}{f(e_{n-1} + x) - f(e_{n-2} + x)} - x
\end{displaymath}

Simplificando:

\begin{equation}
    \label{eq7_regula_falsi}
    e_n = \frac{(e_{n-2})(f(e_{n-1} + x))-(e_{n-1})(f(e_{n-2} + x))}{f(e_{n-1} + x) - f(e_{n-2} + x)}
\end{equation}

\subsection{Expandiendo por las series de Taylor:}

La expansión de Taylor es dada por:

\begin{displaymath}
    [f(a + h) = f(a) + hf^\prime(a) + \frac{h^2}{2}f^{\prime\prime}(a) + \ldots]
\end{displaymath}

Con ella obtenemos:

\begin{displaymath}
    \begin{split}
        &e_n - \frac{e_{n-2}[f(x) + e_{n-1}f^\prime (x) + (\frac{e_{n-1}^2}{2!}) f^{\prime\prime} (x) \ldots]}{[f(x) + e_{n-1}f^\prime (x) + \frac{e_{n-1}^2}{2! f^{\prime\prime} (x)}] - [f(x) + e_{n-2}f^\prime (x) + \frac{e_{n-2}^2}{2! f^{\prime\prime} (x)}]}\\
        &\quad - \frac{e_{n-1}[f(x) + e_{n-2}f^\prime (x) + (\frac{e_{n-2}^2}{2!}) f^{\prime\prime} (x) \ldots]}{[f(x) + e_{n-1}f^\prime (x) + \frac{e_{n-1}^2}{2! f^{\prime\prime} (x)}] - [f(x) + e_{n-2}f^\prime (x) + \frac{e_{n-2}^2}{2! f^{\prime\prime} (x)}]}
    \end{split}
\end{displaymath}

Efectuando:

\begin{displaymath}
    e_n = \frac{e_{n-1} e_{n-2}}{2} \frac{f^{\prime\prime} (x)}{f^\prime (x)}
\end{displaymath}

\begin{displaymath}
    e_n = \frac{1}{2} \frac{f^{\prime\prime} (x)}{f^\prime (x)} e_{n-1} e_{n-2}
\end{displaymath}

\begin{equation}
    \label{eq8_regula_falsi}
    e_n = C e_{n-1} e_{n-2}
\end{equation}

Donde $ C = \frac{1}{2} \frac{f^{\prime\prime} (x)}{f^\prime (x)} $ es la constante del error asintótico.\newline

Debido que Falsa posición se fija a un punto, por ejempo, podemos asumir $ Ce_{n-2} $ es una constante $ K $, por lo tanto, podemos reescribir \eqref{eq8_regula_falsi} como:

\begin{equation}
    \label{eq9_regula_falsi}
    e_n = K e_{n-1}
\end{equation}

Por lo tanto, de la ecuación \eqref{eq9_regula_falsi} se puede deducir que la falsa posición tiene una tasa de convergencia lineal cuando $ |K| < 1 $.
\section{Ventajas y Desventajas del método}

\subsection{Ventajas}

\begin{itemize}
    \item Es un método horquillado \footnote{Los métodos de horquillado determinan intervalos cada vez más pequeños (horquillas) que contienen una raíz.} y siempre es convergente.
    \item El error es controlable, a mayor número de iteraciones la raíz es más precisa.
    \item No requiere el cálculo de derivadas.
    \item No es necesario tener información de la función, aparte del signo de las imágenes.

\end{itemize}

\subsection{Desventajas}

\begin{itemize}
    \item La tasa de convergencia es lenta.
    \item Tiene una tasa de convergencia lineal.
    \item No logra determinar raíces complejas.
    \item No logra identificar múltiples raíces diferentes.
    \item No se puede aplicar en un intervalo donde la función toma valores del mismo signo.

\end{itemize}
\section{Pseudocódigo}

Para encontrar una solución para $ f(x) = 0 $ dada la función $ f $ continua en el intervalo $ (p_0, p_1) $ donde $ f(p_0) $ y $ f(p_1) $ tienen signos opuestos:

\begin{algorithm}
    \caption{Método de posición falsa}
    \KwIn{Aproximación inicial $ p_0, p_1 $;  tolerancia $ TOL $;  máximo número de iteraciones $ N_0 $.}
    \KwOut{Solución aproximada $ p $ o mensaje de error.}
    
    $ i = 2 $; $ q_0 = f(p_0) $; $ q_1 = f(p_1) $
    
    \BlankLine
    
    \While {$i \le N_0 $} {
        $ p = p_0 - \frac{q_0 (p_1 - p_0)}{q_1 - q_0} $
        
        \BlankLine
        
        \If{$ |p - p_0| < TOL $} {
                \tcp {Procedimiento exitoso}
                \KwOut {$ p $}
            }
        
        \BlankLine
        
        $ i = i + 1 $; $ q = f(p) $
        \BlankLine
        
        \If{$ q . q_1 < 0 $} {
                $ p_0 = p_1 $; $ q_0 = q_1 $
            }
            
        \BlankLine
        
        $ p_0 = p $; $ q_0 = q $
    }
    
    \KwOut{El método falló luego de $ N_0 $ iteraciones}
\end{algorithm}
\section{Ejemplos}

\subsection{Dato curioso}
El método de la falsa posición se aplica a la predicción de cantidades traza de contaminantes atmosféricos producidos por reacciones de combustión, como las que se encuentran en fuentes puntuales industriales.

\subsection{Ejercicio}
Considerando la función:
\begin{displaymath}
    f(x) = 1 + 2x - 3x^2e^{-x} + 2x^3sin(x)e^{- \frac{x}{5}}
\end{displaymath}
Calcular el cruce por cero dentro del intervalo $ [6, 7] $; usando un error máximo de $ 10^{–5} $.

\begin{table}[h!]
    \centering
    \caption{Resultados del cálculo de los cruces por cero.}
    \label{table1_regula_falsi}
    \begin{tabular}{c|c|c|c|c|c|c}
        \textbf{$ n $} & \textbf{$ x_{n-1} $} & \textbf{$ f(x_{n-1}) $} & \textbf{$ x_n $} & \textbf{$ f(x_n) $} & \textbf{$ x_{n+1} $} & \textbf{$ f(x_{n+1}) $}\\
        \hline
         $ 0 $ & $ 6.0000 $ & $ -23.624 $ & $ 7.0000 $ & $ 126.00 $ & $ 6.1578 $ & $ -3.9574 $ \\
         $ 1 $ & $ 6.1578 $ & $ -3.9574 $ & $ 7.0000 $ & $ 126.00 $ & $ 6.1835 $ & $ -0.5292 $ \\ 
         $ 2 $ & $ 6.1835 $ & $ -0.5292 $ & $ 7.0000 $ & $ 126.00 $ & $ 6.1869 $ & $ -0.0681 $ \\ 
         $ 3 $ & $ 6.1869 $ & $ -0.0681 $ & $ 7.0000 $ & $ 126.00 $ & $ 6.1873 $ & $ -0.0087 $ \\ 
         $ 4 $ & $ 6.1873 $ & $ -0.0087 $ & $ 7.0000 $ & $ 126.00 $ & $ 6.1874 $ & $ -0.0011 $ \\ 
         $ 5 $ & $ 6.1874 $ & $ -0.0011 $ & $ 7.0000 $ & $ 126.00 $ & $ 6.1874 $ & $ -0.0001 $ \\ 
         $ 6 $ & $ 6.1874 $ & $ -0.0001 $ & $ 7.0000 $ & $ 126.00 $ & $ 6.1874 $ & $ -1.8e-5 $ \\ 
    \end{tabular}
\end{table}

El cruce por cero de la función $ f(x) $ en $ [6, 7] $ nos devuelve un valor de $ x_{n+1} = 6.1874$

\begin{figure}[!ht]
    \centering
    \begin{tikzpicture}
        \begin{axis}[
            axis x line = middle,
            axis y line =  center,
            grid = minor
            ]
            \addplot[
                color = red,
                domain = 5:10
                ]{1 + 2*x - 3*(x^2)*pow(e,-x) + 2*(x^3)*sin(deg(x))*pow(e,-x/5)};
            \filldraw[black] (6.1874, -0.000018) circle (2pt) node[anchor=south west]{$ (6.1874, -0.000018) $};
        \end{axis}
    \end{tikzpicture}
    \caption{Gráfica de $ f(x) = 1 + 2x - 3x^2e^{-x} + 2x^3sin(x)e^{- \frac{x}{5}} $}
    \label{regula_grap_ej2}
\end{figure}

\newpage

\bibliographystyle{plain}
\bibliography{bibliography.bib}

\end{document}