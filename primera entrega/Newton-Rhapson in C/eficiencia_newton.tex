\section{Análisis sobre la eficiencia del método de Newton-Raphson}

El método de Newton-Raphson, tiene una convergencia cuádratica, como se demostrará partiendo del siguiente teorema

\begin{theorem}
Asumiendo que $f,f',f''  $sean continuas para toda x en las proximidades de $p$ y asumiendo que $f(p) = 0, f'(p) \neq 0 $.Entonces si se escoge lo suficientemente cerca de $p$ , las iteraciones del método de Newton Raphson convergen a $p$. Con un orden de convergencia cuadrático
\end{theorem}

\begin{proof}
Partiendo de 

\begin{displaymath}
\lim_{n->\infty}{\frac{p_{n+1} - p}{p_n - p}} = \lambda
\end{displaymath}
Usaremos el polinomio de Taylor de $f(x)$ en \textit{p}
\begin{displaymath}
 0 = f(p_n) + (p-p_n)f'(p_n) + \frac{(p-p_n)^2}{2}f''(p)
\end{displaymath}
obtenemos
\begin{displaymath}
p - p_{n+1} =  - (p-p_n)^2 \frac{f''(p)}{2f'(p)}
\end{displaymath}
\begin{displaymath}
\frac{|p - p_{n+1}|}{|(p-p_n)^2|} = \frac{f''(p)}{2f'(p)}
\end{displaymath}

\begin{displaymath}
\lim_{n->\infty}{\frac{|p - p_{n+1}|}{|(p-p_n=|)^2}} =  \frac{f''(p)}{2f'(p)}
\end{displaymath}
Lo que prueba que la convergencia del método es cuadrática
\end{proof}

